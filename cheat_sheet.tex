\chapter{Cheat Sheet}\label{chapter:cheat_sheet}

Hier findet Ihr einen Überblick über die wichtigsten \LaTeX\ Befehle.

\textbf{Fetter Text} \\
\textit{Kursiver Text} \\
\underline{Unterstrichener Text} \\

\section{Einfügen von Aufzählungen}\label{section:aufzaehlungen}

    \begin{itemize}
        \item Erstes Element
        \item Zweites Element
        \item Drittes Element
    \end{itemize}

\section{Einfügen von Bildern}\label{section:bilder}

    % Einfügen einer Grafik
    \begin{figure}[htb]
        % Zentrierung
        \centering
        % Einfügen der Datei, mit angepasster Höhe
        \includegraphics[height=5cm]{graphics/dhbw.png}
        % Titel und Label der Grafik
        \caption[Logo der DHBW]{Logo der DHBW.\footnotemark}
        \label{abb:DHBWLogo}
    \end{figure}
    \footnotetext{Logo der DHBW}

\section{Einfügen von Tabellen}\label{section:tabellen}

% Einfügen einer Tabelle
\begin{table}[htb]
    % Zentrierung
    \centering
    % Beginn der eigentlichen Tabelle
    % Im zweiten Klammerpaar, Definition des Tabellendesign
    % Ausrichtung (l, c, r) und vertikale Trennung (|)
    \begin{tabular}{lcr}
        % Beginn einer neuen Spalte mit & und einer neuen Zeile mit \\
        links & Mitte & rechts \\
        % Einfügen einer horizontalen Linie
        \hline
        Muster & Muster & Muster \\
    \end{tabular}
    % Titel und Label der Tabelle
    \caption{Kleine Beispiel-Tabelle.}
    \label{tab:BeispielTabelleKlein}
\end{table}

\section{Einfügen von Source Code}\label{section:source_code}

% Definition, welche Programmiersprache verwendet wird
\lstset{language=Java}

    In \LaTeX\ kann auch Source Code referenziert werden.

    Dieser kann direkt in das Dokument geschrieben werden:

    % Beginn des Code Blocks
    \begin{lstlisting}
        if(1 > 0) {
        System.out.println("OK"); 
        } else {
        System.out.println("merkwuerdig");
        }
    \end{lstlisting}

\section{Einfügen von Abkürzungen}\label{section:abkuerzungen}

Beim ersten mal wird die Abkürzung ausgeschrieben: \ac{DHBW}

Bei allen weiteren Verwendungen wird nur die Abkürzung mit Link dargestellt: \ac{DHBW}

\section{Einfügen von Fußnoten und Zitaten}\label{section:zitate}

Fußnoten können auf verschiedene Weisen in \LaTeX eingefügt werden.

Hier wird nur eine beispielhafte Variante gezeigt!

    \begin{itemize}
        \item 'Direktes Zitate'\footnote{\cite[S. 114ff.]{Mayring2002}}
        \item Indirektes Zitate\footnote{Vgl. \cite[S. 114ff.]{Mayring2002}}
        \item Sekundäres indirektes Zitat\footnote{Vgl. \cite[S. 114ff.]{Mayring2002} nach \cite{Endres}}
    \end{itemize}

\section{Einfügen von Referenzen}\label{section:referenzen}

Auf Inhalte kann referenziert werden. Hierzu müssen diese Inhalte nur mit einem Label versehen werden.

    \begin{itemize}
        \item Bilder\ref{abb:DHBWLogo}
        \item Kapitel\ref{chapter:cheat_sheet}
        \item Abschnitte\ref{section:bilder}
    \end{itemize}

