%%% Präambel %%%
% hier sollten keine Änderungen erforderlich sein
%
\usepackage[utf8]{inputenc}   % Zeichencodierung UTF-8 für Eingabe-Dateien
\usepackage[T1]{fontenc}      % Darstellung von Umlauten im PDF

\usepackage{ifthen}

\usepackage{listings}         % für Einbindung von Code-Listings
\lstset{numbers=left,numberstyle=\tiny,numbersep=5pt,texcl=true}
\lstset{literate=             % erlaubt Sonderzeichen in Code-Listings 
  {Ö}{{\"O}}1
{Ä}{{\"A}}1
{Ü}{{\"U}}1
{ß}{{\ss}}2
{ü}{{\"u}}1
{ä}{{\"a}}1
{ö}{{\"o}}1
{€}{{\euro}}1
}

\usepackage[
  inner=35mm,outer=15mm,top=25mm,
  bottom=20mm,foot=12mm,includefoot
]{geometry}                 % Einstellungen für Ränder

\usepackage[ngerman]{babel} % Spracheinstellungen Deutsch
\usepackage[babel,german=quotes]{csquotes} % deutsche Anf.zeichen
\usepackage{enumerate}      % anpassbare Nummerier./Aufz.
\usepackage{graphicx}       % Einbinden von Grafiken
\usepackage[onehalfspacing]{setspace} % anderthalbzeilig

\usepackage{blindtext}      % Textgenerierung für Testzwecke
\usepackage{color}          % Verwendung von Farbe 

\usepackage{acronym}        % für ein Abkürzungsverzeichnis

\usepackage[                % Biblatex
  backend=biber,
  bibstyle=lib/template/_dhbw_authoryear,maxbibnames=99,
  citestyle=authoryear,
  uniquename=true, useprefix=true,
  bibencoding=utf8]{biblatex}
%kein Punkt am Ende bei \footcite
%http://www.golatex.de/footcite-ohne-punkt-am-schluss-t4865.html
\renewcommand{\bibfootnotewrapper}[1]{\bibsentence#1}


%Reihenfolge der Autorennamen
%   
% http://golatex.de/viewtopic,p,80448.html#80448
% Argumente: siehe http://texwelt.de/blog/modifizieren-eines-biblatex-stils/
\DeclareNameFormat{sortname}{% Bibliographie
  \ifnum\value{uniquename}=0 % Normalfall
    \ifuseprefix%
      {%
         \usebibmacro{name:family-given}
           {\namepartfamily}
           {\namepartgiveni}
           {\namepartprefix}
           {\namepartsuffixi}%
       }
      {%
         \usebibmacro{name:family-given}
           {\namepartfamily}
           {\namepartgiveni}
           {\namepartprefixi}
           {\namepartsuffixi}%
       }%
  \fi
  \ifnum\value{uniquename}=1% falls nicht eindeutig, abgek. Vorname 
      {%
         \usebibmacro{name:family-given}
           {\namepartfamily}
           {\namepartgiveni}
           {\namepartprefix}
           {\namepartsuffix}%
       }%
  \fi
  \ifnum\value{uniquename}=2% falls nicht eindeutig, ganzer Vorname 
      {%
         \usebibmacro{name:family-given}
           {\namepartfamily}
           {\namepartgiven}
           {\namepartprefix}
           {\namepartsuffix}%
       }%
  \fi   
  \usebibmacro{name:andothers}}

\DeclareNameFormat{labelname}{% für Zitate
  \ifnum\value{uniquename}=0 % Normalfall
    \ifuseprefix%
      {%
         \usebibmacro{name:family-given}
           {\namepartfamily}
           {\empty}
           {\namepartprefix}
           {\namepartsuffixi}%
       }
      {%
         \usebibmacro{name:family-given}
           {\namepartfamily}
           {\empty}
           {\namepartprefixi}
           {\namepartsuffixi}%
       }%
  \fi
  \ifnum\value{uniquename}=1% falls nicht eindeutig, abgek. Vorname 
      {%
         \usebibmacro{name:family-given}
           {\namepartfamily}
           {\namepartgiveni}
           {\namepartprefix}
           {\namepartsuffix}%
       }%
  \fi
  \ifnum\value{uniquename}=2% falls nicht eindeutig, ganzer Vorname 
      {%
         \usebibmacro{name:family-given}
           {\namepartfamily}
           {\namepartgiven}
           {\namepartprefix}
           {\namepartsuffix}%
       }%
  \fi   
  \usebibmacro{name:andothers}}
      
  
\DeclareFieldFormat{extrayear}{% = the 'a' in 'Jones 1995a'
  \iffieldnums{labelyear}
    {\mknumalph{#1}}
    {\mknumalph{#1}}}        

\renewcommand*{\multinamedelim}{\addslash}
\renewcommand*{\finalnamedelim}{\addslash}
\renewcommand*{\multilistdelim}{\addslash}
\renewcommand*{\finallistdelim}{\addslash}

\renewcommand{\nameyeardelim}{~}

% Literaturverzeichnis: Doppelpunkt zwischen Name (Jahr): Rest 
% http://de.comp.text.tex.narkive.com/Tn1HUIXB/biblatex-authoryear-und-doppelpunkt
\renewcommand{\labelnamepunct}{\addcolon\addspace}

% damit die Darstellung für Vollzitate von Primärquellen in 
% Fußnoten später auf "nicht fett" geändert werden kann 
% (nur für Zitate von Sekundärliteratur relevant)
\newcommand{\textfett}[1]{\textbf{#1}}

% für Zitate von Sekundärliteratur:
\newcommand{\footcitePrimaerSekundaer}[4]{%
  \renewcommand{\textfett}[1]{##1}%
  \footnote{\fullcite[#2]{#1}, zitiert nach \cite[#4]{#3}}%  
  \renewcommand{\textfett}[1]{\textbf{##1}}%
}

% Im Literaturverzeichnis: Autor (Jahr) fett
\renewbibmacro*{author}{%
  \ifboolexpr{%
    test \ifuseauthor%
    and
    not test {\ifnameundef{author}}
  }
    {\usebibmacro{bbx:dashcheck}
       {\bibnamedash}
       {\usebibmacro{bbx:savehash}%
        \textfett{\printnames{author}}%
        \iffieldundef{authortype}
          {\setunit{\addspace}}
          {\setunit{\addcomma\space}}}%
     \iffieldundef{authortype}
       {}
       {\usebibmacro{authorstrg}%
        \setunit{\addspace}}}%
    {\global\undef\bbx@lasthash
     \usebibmacro{labeltitle}%
     \setunit*{\addspace}}%
  \textfett{\usebibmacro{date+extrayear}}}

% Sonderfall: Quelle ohne Autor, aber mit Herausgeber
% Name des Herausgebers wird fett gedruckt
\renewbibmacro*{bbx:editor}[1]{%
  \ifboolexpr{%
    test \ifuseeditor%
    and
    not test {\ifnameundef{editor}}
  }
    {\usebibmacro{bbx:dashcheck}
       {\bibnamedash}
       {\textfett{\printnames{editor}}%
        \setunit{\addcomma\space}%
        \usebibmacro{bbx:savehash}}%
     \usebibmacro{#1}%
     \clearname{editor}%
     \setunit{\addspace}}%
    {\global\undef\bbx@lasthash
     \usebibmacro{labeltitle}%
     \setunit*{\addspace}}%
  \textfett{\usebibmacro{date+extrayear}}}

% Anpassungen für deutsche Sprache
\DefineBibliographyStrings{ngerman}{%
	nodate = {{o.J.}},
	urlseen = {{Abruf:}},
	ibidem = {{ebenda}}
}

% keine Anführungszeichen beim Titel im Literaturverzeichnis
\DeclareFieldFormat[article,book,inbook,inproceedings,manual,misc,phdthesis,thesis,online,report]{title}{#1\isdot}

\newcommand{\literaturverzeichnis}{%
% nur Literaturverzeichnis
% (als eigenes Kapitel)
\phantomsection
\addcontentsline{toc}{chapter}{Literaturverzeichnis}
\spezialkopfzeile{Literaturverzeichnis}
\defbibheading{lit}{\chapter*{Literaturverzeichnis}}
\label{chapter:quellen}
\printbibliography[heading=lit,notkeyword=ausblenden]
} % mit DHBW-spezifischen Einstellungen

\usepackage{hyperref}       % URL-Formatierung, klickbare Verweise

\usepackage{tocloft}        % für Verzeichnis der Anhänge

\newcounter{anhcnt}
\setcounter{anhcnt}{0}
\newlistof{anhang}{app}{}

\newcommand{\anhang}[1]{%
  \refstepcounter{anhcnt}
  \setcounter{anhteilcnt}{0}
  \section*{Anhang \theanhcnt: #1}
  \addcontentsline{app}{section}{\protect\numberline{Anhang \theanhcnt}#1}\par
}

\newcounter{anhteilcnt}
\setcounter{anhteilcnt}{0}

\newcommand{\anhangteil}[1]{%
  \refstepcounter{anhteilcnt}
  \subsection*{Anhang~\arabic{anhcnt}/\arabic{anhteilcnt}: #1}
  \addcontentsline{app}{subsection}{\protect\numberline{Anhang \theanhcnt/\arabic{anhteilcnt}}#1}\par
}

\renewcommand{\theanhteilcnt}{Anhang \theanhcnt/\arabic{anhteilcnt}}

% vgl. S. 4 Paket-Beschreibung tocloft 	
% Einrückungen für Anhangverzeichnis
\makeatletter
\newcommand{\abstaendeanhangverzeichnis}{
  \renewcommand*{\l@section}{\@dottedtocline{1}{0em}{5.5em}}
  \renewcommand*{\l@subsection}{\@dottedtocline{2}{2.3em}{6.5em}}
}
\makeatother

% Abbildungs- und Tabellenverzeichnis
% Bezeichnungen
\renewcaptionname{ngerman}{\figurename}{Abb.}
\renewcaptionname{ngerman}{\tablename}{Tab.}
% Einrückungen
\makeatletter
\renewcommand*{\l@figure}{\@dottedtocline{1}{0em}{2.3em}}
\renewcommand*{\l@table}{\@dottedtocline{1}{0em}{2.3em}}
\makeatother


\usepackage{chngcntr}                % fortlaufende Zähler für Fußnoten, Abbildungen und Tabellen
\counterwithout{figure}{chapter}
\counterwithout{table}{chapter}
\counterwithout{footnote}{chapter}

\usepackage[automark]{scrlayer-scrpage}
%% Definitionen für Kopf- und Fußzeile auf normalen Seiten
\defpagestyle{kopfzeile}
{% Kopfdefinition
  (\textwidth,0pt)    % Länge der oberen Linie,Dicke der oberen Linie       
  {} % Definition für linke Seiten im doppelseitigen Layout
  {} % Definition für rechte Seiten im doppelseitigen Layout      
  {  % Definition für Seiten im einseitigen Layout
	\makebox[0pt][l]{\rightmark}% 
	\makebox[\linewidth]{}% 
  }        
  (\textwidth, 0.4pt) % Untere Linienlänge, Untere Liniendicke
}
{% Fußdefinition
  (\textwidth,0pt)    % Obere Linienlänge, Obere Liniendicke
  {} % Definition für linke Seiten im doppelseitigen Layout
  {} % Definition für rechte Seiten im doppelseitigen Layout
  {  % Definition für Seiten im einseitigen Layout
    \makebox[\linewidth]{}%
    \makebox[0pt][r]{\pagemark}%
  }
  (\textwidth, 0pt)   % Länge der unteren Linie,Dicke der unteren Linie
}

%% Definitionen für Kopf- und Fußzeile auf ersten Seiten eines Kapitels
\defpagestyle{kapitelkopfzeile}
{% Kopfdefinition
  (\textwidth,0pt)    % Länge der oberen Linie,Dicke der oberen Linie       
  {} % Definition für linke Seiten im doppelseitigen Layout
  {} % Definition für rechte Seiten im doppelseitigen Layout      
  {}  % Definition für Seiten im einseitigen Layout
  (\textwidth, 0pt) % Untere Linienlänge, Untere Liniendicke
}
{% Fußdefinition
  (\textwidth,0pt)    % Obere Linienlänge, Obere Liniendicke
  {} % Definition für linke Seiten im doppelseitigen Layout
  {} % Definition für rechte Seiten im doppelseitigen Layout
  {  % Definition für Seiten im einseitigen Layout
    \makebox[\linewidth]{}%
    \makebox[0pt][r]{\pagemark}%
  }
  (\textwidth, 0pt)   % Länge der unteren Linie,Dicke der unteren Linie
}

%% Definitionen für Kopf- und Fußzeile im Anhang und bei Quellenverzeichnisse
\newcommand{\spezialkopfzeileBezeichnung}{}
\defpagestyle{spezialkopfzeile}
{% Kopfdefinition
  (\textwidth,0pt)    % Länge der oberen Linie,Dicke der oberen Linie       
  {} % Definition für linke Seiten im doppelseitigen Layout
  {} % Definition für rechte Seiten im doppelseitigen Layout      
  {  % Definition für Seiten im einseitigen Layout
	\makebox[0pt][l]{\spezialkopfzeileBezeichnung}% 
	\makebox[\linewidth]{}% 
  }        
  (\textwidth, 0.4pt) % Untere Linienlänge, Untere Liniendicke
}
{% Fußdefinition
  (\textwidth,0pt)    % Obere Linienlänge, Obere Liniendicke
  {} % Definition für linke Seiten im doppelseitigen Layout
  {} % Definition für rechte Seiten im doppelseitigen Layout
  {  % Definition für Seiten im einseitigen Layout
    \makebox[\linewidth]{}%
    \makebox[0pt][r]{\pagemark}%
  }
  (\textwidth, 0pt)   % Länge der unteren Linie,Dicke der unteren Linie
}
            
\newcommand\spezialkopfzeile[1]{%
  \renewcommand\spezialkopfzeileBezeichnung{#1}
  \pagestyle{spezialkopfzeile}
}
                
% Standard-Pagestyle auswählen
\pagestyle{kopfzeile}

% keine Kopfzeile anzeigen auf Seiten, auf denen ein 
% Kapitel beginnt oder das Inhalts-/Abbildungs-/Tabellenverzeichnis steht 
\renewcommand{\chapterpagestyle}{kapitelkopfzeile}
\tocloftpagestyle{kapitelkopfzeile}

		 % für schöne Kopfzeilen 

\usepackage{textcomp}            % erlaubt EUR-Zeichen in Eingabedatei
\usepackage{eurosym}             % offizielles EUR-Symbol in Ausgabe
\renewcommand{\texteuro}{\euro}  % ACHTUNG: nach hyperref aufrufen!

\usepackage{scrhack}             % stellt Kompatibilität zw. KOMA-Script
% (scrreprt) und anderen Paketen her

% Anpassung der Abstände bei Kapitelüberschriften
% (betrifft auch Inhalts-, Abbildungs- und Tabellenverzeichnis)
\renewcommand*\chapterheadstartvskip{\vspace*{-\topskip}}
\newcommand{\myBeforeTitleSkip}{1mm}
\newcommand{\myAfterTitleSkip}{10mm}
\setlength\cftbeforetoctitleskip{\myBeforeTitleSkip}
\setlength\cftbeforeloftitleskip{\myBeforeTitleSkip}
\setlength\cftbeforelottitleskip{\myBeforeTitleSkip}

\setlength\cftaftertoctitleskip{\myAfterTitleSkip}
\setlength\cftafterloftitleskip{\myAfterTitleSkip}
\setlength\cftafterlottitleskip{\myAfterTitleSkip}
%%% Ende der Präambel %%%
