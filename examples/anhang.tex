\chapter*{Anhang}
\addcontentsline{toc}{chapter}{Anhang}
\section*{Anhangverzeichnis}
\vspace{-8em}

% vor \listofanhang müssen Einrückungen angepasst werden
\abstaendeanhangverzeichnis

\listofanhang
\clearpage
\spezialkopfzeile{Anhang} % damit in der Kopfzeile das Wort "Anhang" angezeigt wird

\anhang{So funktioniert's}

\lstset{language=TeX, % hervorzuhebende Keywords definieren
  morekeywords={anhang, anhangteil}
}


Um den Anforderungen der Zitierrichtlinien nachzukommen, wird das Paket \verb|tocloft| verwendet. Jeder Anhang wird mit dem (neu definierten) Befehl \lstinline|\anhang{Bezeichnung}| begonnen, der insbesondere dafür sorgt, dass ein Eintrag im Anhangsverzeichnis erzeugt wird. Manchmal ist es wünschenswert, auch einen Anhang noch weiter zu unterteilen. Hierfür wurde der Befehl \lstinline|\anhangteil{Bezeichnung}| definiert.

In~\ref{anhang:abbildung} finden Sie eine bekannte Abbildung und etwas Source Code in~\ref{anhang:sourcecode}. 

\anhangteil{Wieder mal eine Abbildung}\label{anhang:abbildung}
\begin{figure}[htb]
\centering
\includegraphics[width=0.9\linewidth]{graphics/dhbw.png}
\caption{Mal wieder das DHBW-Logo.}
\end{figure}

\anhangteil{Etwas Source Code}\label{anhang:sourcecode}
\lstinputlisting{includes/HelloWorld.java}

