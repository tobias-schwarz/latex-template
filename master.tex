% Grundlegende Dokumenteneigenschaften gemäß DHBW-Vorgaben
\documentclass[a4paper,fontsize=11pt,oneside,parskip=half,headings=normal]{scrreprt} 
% \usepackage{showframe} % nur für Kontrolle der Ränder 

%%% Präambel einbinden (mit Festlegungen gemäß DHBW-Vorgaben) %%%
\input{lib/template/_dhbw_praeambel.tex}

%%% Name der eigenen Literatur-Datenbank %%%
\bibliography{lib/bibliography.bib}

\begin{document}

%%% Deckblatt einbinden %%% 
% Typ der Arbeit (für Deckblatt und ehrenwörtliche Erklärung) - bitte Zutreffendes auswählen
% \newcommand{\typMeinerArbeit}{1. Projektarbeit} 
% \newcommand{\typMeinerArbeit}{2. Projektarbeit} 
% \newcommand{\typMeinerArbeit}{Seminararbeit} 
% \newcommand{\typMeinerArbeit}{Bachelorarbeit} 
\newcommand{\typMeinerArbeit}{Benutzerdefinierter Typ}

% Thema der Arbeit (für ehrenwörtliche Erklärung, ohne Umbrüche)
\newcommand{\themaMeinerArbeit}{Mein Titel}

% Ggf. Untertitel der Arbeit (falls dies nicht benötigt wird, das zweite Klammerpaar leer lassen, nicht das Kommando löschen!)
\newcommand{\unterThemaMeinerArbeit}{Untertitel}

% Vorname, Name der Autorin/des Autors (für Titelseite und Metadaten)
\newcommand{\meinName}{Vorname Nachname}

\thispagestyle{empty}

\begin{spacing}{1}
\begin{center}	
~\vspace{0mm}

% Titel der Arbeit
{\sffamily
    \LARGE  
    \textbf{\themaMeinerArbeit}
    
    \bigskip
    \textbf{\unterThemaMeinerArbeit}
}

\vspace{15mm}

% Typ der Arbeit
{\Large \typMeinerArbeit}

\vspace{1cm}

% Datum festlegen
vorgelegt am \today 

\vspace{15mm}

% Fakultät festlegen
Fakultät Wirtschaft
\medskip

% Studiengang festlegen
Studiengang Wirtschaftsinformatik
\medskip

% Kurs festlegen
Kurs ... 

\vspace{10mm}

von

\vspace{10mm}

% Autor/in/en/innen festlegen
{\large\textsc{\meinName}}

\vspace{10mm}
\end{center}

\vfill

% HIER EDITIEREN: Name des Unternehmens, Name der Betreuerin/des Betreuers
\begin{tabular}{ll}
Betreuer*in in der Ausbildungsstätte: & DHBW Stuttgart: \\
\hspace{0.4\linewidth} & \\
Name des Unternehmens & Titel, Vorname und Nachname \\
Titel, Vorname und Nachname der Betreuerin & der/des wissenschaftlichen Betreuerin/Prüferin \\
Funktion der Betreuerin/des Betreuers \\
\\
Unterschrift der Betreuerin/des Betreuers \\
\end{tabular}


\vspace{1cm}
%(etwas Platz für die Unterschrift der Betreuerin/des Betreuers aus der Ausbildungsstätte)
\end{spacing}

% falls ein Vertraulichkeitsvermerk erforderlich ist,
% die Kommentarzeichen in den nachfolgenden Zeilen entfernen:
 
%\begin{center}
%\small
%\textbf{Vertraulichkeitsvermerk}:
%Der Inhalt dieser Arbeit darf weder als Ganzes noch in Auszügen \\
%Personen außerhalb des Prüfungs- und Evaluationsverfahrens zugänglich gemacht werden, sofern keine anders lautende Genehmigung des Dualen Partners vorliegt. 
%\end{center}

% Meta-Daten für PDF-Datei basierend auf obigen Angaben
\hypersetup{pdftitle={\themaMeinerArbeit}}
\hypersetup{pdfauthor={\meinName}}
\hypersetup{pdfsubject={\typMeinerArbeit\ DHBW Stuttgart \the\year}}


%%% Umstellung der Seiten-Nummerierung auf i, ii, iii ... %%%
\pagenumbering{Roman}

%%% Abstract einbinden (optionale Kurzfassung Ihrer Arbeit) %%%
% \begin{abstract}
\thispagestyle{kapitelkopfzeile}
\textbf{\LaTeX-Vorlage für Projekt-, Seminar- und Bachelorarbeiten}

Bei dem vorliegenden Dokument handelt es sich um eine Vorlage, die
für Projekt-, Seminar- und Bachelorarbeiten im Studiengang
Wirtschaftsinformatik der DHBW Stuttgart verwendet werden kann.

Sie setzt die technischen Vorgaben der Zitierrichtlinien\footnote{Sie finden diese unter \enquote{Prüfungsleistungen} im Studierendenportal (\url{https://studium.dhbw-stuttgart.de/winf/pruefungsleistungen/}).} des Studiengangs
(Stand: 01/2020) um.

\emph{Hinweise:} Bitte lesen Sie sich die Zitierrichtlinien unbedingt genau durch. Dieses Dokument ersetzt keine Anleitung oder Einführung in \LaTeX,
für die Nutzung sind daher gewisse Vorkenntnisse unerlässlich. Ein Einstieg in 
\LaTeX\ ist aber weniger schwierig, als es vielleicht auf den ersten Blick scheint
und lohnt sich für das Verfassen wissenschaftlicher Arbeiten in jedem Fall.\footnote{%
so auch \url{http://www.spiegel.de/netzwelt/tech/textsatz-keine-angst-vor-latex-a-549509.html}} 
Als Hilfestellung beim Schreiben eines Dokuments habe ich einen zweiseitigen kompakten \LaTeX-Spickzettel erstellt, der über Moodle verfügbar ist.

Ihre Rückmeldungen und Anregungen zu dieser Vorlage nehme ich gerne per E-Mail an die Adresse
\url{tobias.straub@dhbw-stuttgart.de} entgegen.

--- Prof. Dr. Tobias Straub

\vspace{5em}

\begin{center}\small
\begin{tabular}{ccl}
\multicolumn{3}{c}{\textbf{Versionshistorie}}\\
\hline
1.0	& 05.02.2015 & erste Fassung \\
\hline
1.1 & 16.02.2015 & siehe~\ref{anhang:ReleaseNotes11} \\
\hline
1.2 & 20.04.2015 & siehe~\ref{anhang:ReleaseNotes12} \\
\hline
1.3 & 20.02.2016 & siehe~\ref{anhang:ReleaseNotes13} \\
\hline
1.4 & 24.07.2017 & siehe~\ref{anhang:ReleaseNotes14} \\
\hline
1.5 & 07.01.2018 & siehe~\ref{anhang:ReleaseNotes15} \\
\hline
1.6 & 07.04.2018 & siehe~\ref{anhang:ReleaseNotes16} \\
\hline
1.7 & 12.02.2019 & siehe~\ref{anhang:ReleaseNotes17} \\
\hline
1.8 & 10.02.2020 & siehe~\ref{anhang:ReleaseNotes18} \\
\end{tabular}
\end{center}

\end{abstract}


% \cleardoublepage

\begin{spacing}{1}

  %%% Inhaltsverzeichnis %%%
  \tableofcontents
  \clearpage

  %%% Abkürzungsverzeichnis %%%
  \chapter*{Abkürzungsverzeichnis}
\addcontentsline{toc}{chapter}{Abkürzungsverzeichnis}

\begin{acronym}[DHBW] 
% Argument definiert die Breite der ersten Spalte anhand des längsten vorkommenden Eintrags
\acro{CRM}{Customer Relationship Management}
\acro{DHBW}{Duale Hochschule Baden-Württemberg}
\end{acronym}

  \clearpage

  \thispagestyle{kapitelkopfzeile}

  %%% Abbildungsverzeichnis %%%
  \listoffigures
  \phantomsection

  %%% Einfügen des Abbildungsverzeichnisses in das Inhaltsverzeichnis %%%
  \addcontentsline{toc}{chapter}{Abbildungsverzeichnis}
  \clearpage

  %%% Tabellenverzeichnis %%%
  \listoftables
  \phantomsection

  %%% Einfügen des Tabellenverzeichnisses in das Inhaltsverzeichnis %%%
  \addcontentsline{toc}{chapter}{Tabellenverzeichnis}
  \cleardoublepage

\end{spacing}

%%% Umstellung der Seiten-Nummerierung auf 1, 2, 3 ... %%%
\pagenumbering{arabic}

%%% Beginn des eigentlichen Inhalts %%%
% Empfehlung: strukturieren Sie Ihren Text in einzelnen Dateien und binden Sie diese hier mit   \input{main_part/dateiname.tex} ein

% Cheat Sheet mit einem Überblick über die wichtigsten Befehle
\chapter{Cheat Sheet}\label{chapter:cheat_sheet}

Hier findet Ihr einen Überblick über die wichtigsten \LaTeX\ Befehle.

\textbf{Fetter Text} \\
\textit{Kursiver Text} \\
\underline{Unterstrichener Text} \\

\section{Einfügen von Aufzählungen}\label{section:aufzaehlungen}

    \begin{itemize}
        \item Erstes Element
        \item Zweites Element
        \item Drittes Element
    \end{itemize}

\section{Einfügen von Bildern}\label{section:bilder}

    % Einfügen einer Grafik
    \begin{figure}[htb]
        % Zentrierung
        \centering
        % Einfügen der Datei, mit angepasster Höhe
        \includegraphics[height=5cm]{graphics/dhbw.png}
        % Titel und Label der Grafik
        \caption[Logo der DHBW]{Logo der DHBW.\footnotemark}
        \label{abb:DHBWLogo}
    \end{figure}
    \footnotetext{Logo der DHBW}

\section{Einfügen von Tabellen}\label{section:tabellen}

% Einfügen einer Tabelle
\begin{table}[htb]
    % Zentrierung
    \centering
    % Beginn der eigentlichen Tabelle
    % Im zweiten Klammerpaar, Definition des Tabellendesign
    % Ausrichtung (l, c, r) und vertikale Trennung (|)
    \begin{tabular}{lcr}
        % Beginn einer neuen Spalte mit & und einer neuen Zeile mit \\
        links & Mitte & rechts \\
        % Einfügen einer horizontalen Linie
        \hline
        Muster & Muster & Muster \\
    \end{tabular}
    % Titel und Label der Tabelle
    \caption{Kleine Beispiel-Tabelle.}
    \label{tab:BeispielTabelleKlein}
\end{table}

\section{Einfügen von Source Code}\label{section:source_code}

% Definition, welche Programmiersprache verwendet wird
\lstset{language=Java}

    In \LaTeX\ kann auch Source Code referenziert werden.

    Dieser kann direkt in das Dokument geschrieben werden:

    % Beginn des Code Blocks
    \begin{lstlisting}
        if(1 > 0) {
        System.out.println("OK"); 
        } else {
        System.out.println("merkwuerdig");
        }
    \end{lstlisting}

\section{Einfügen von Abkürzungen}\label{section:abkuerzungen}

Beim ersten mal wird die Abkürzung ausgeschrieben: \ac{DHBW}

Bei allen weiteren Verwendungen wird nur die Abkürzung mit Link dargestellt: \ac{DHBW}

\section{Einfügen von Fußnoten und Zitaten}\label{section:zitate}

Fußnoten können auf verschiedene Weisen in \LaTeX eingefügt werden.

Hier wird nur eine beispielhafte Variante gezeigt!

    \begin{itemize}
        \item 'Direktes Zitate'\footnote{\cite[S. 114ff.]{Mayring2002}}
        \item Indirektes Zitate\footnote{Vgl. \cite[S. 114ff.]{Mayring2002}}
        \item Sekundäres indirektes Zitat\footnote{Vgl. \cite[S. 114ff.]{Mayring2002} nach \cite{Endres}}
    \end{itemize}

\section{Einfügen von Referenzen}\label{section:referenzen}

Auf Inhalte kann referenziert werden. Hierzu müssen diese Inhalte nur mit einem Label versehen werden.

    \begin{itemize}
        \item Bilder\ref{abb:DHBWLogo}
        \item Kapitel\ref{chapter:cheat_sheet}
        \item Abschnitte\ref{section:bilder}
    \end{itemize}


\clearpage

% Beispieldateien
% \input{examples/einleitung.tex}
% \blinddocument
% \input{examples/text_mit_zitaten.tex}
% \input{examples/abbildungen_und_tabellen.tex}

%%% Ende des eigentlichen Inhalts %%%

%%% Beginn des Anhangs %%%
\chapter*{Anhang}
\addcontentsline{toc}{chapter}{Anhang}

\lstset{language=TeX,
    morekeywords={anhang, anhangteil}
}

% Definition des Anhangverzeichnis
\section*{Anhangverzeichnis}
\vspace{-8em}
\abstaendeanhangverzeichnis
\listofanhang
\clearpage

% Konfiguration der speziellen Kopfzeile für den Anhang
\spezialkopfzeile{Anhang}

% Hauptteil des Anhangs
% Mit \anhang{Abschnitt des Anhangs} fügt man ein Kapitel in dem Anhang hinzu z. B. Interview Transkripte
\anhang{Interview Transkripte}

    % Einfügen von weiteren Unterabschnitten, um den Anhang zu gliedern
    % Mit \anhangteil{Name des Unterabschnitts} fügt man weitere Unterabschnitte hinzu
\anhangteil{Interview Transkript: Mitarbeiter eines Unternehmens}

% Es bietet sich an für bestimmte Dinge (wie die Namen von Interviewpartner) Variablen anzulegen
% \newcommand{\Interviewer}{\textbf{Interviewer: }}
% \newcommand{\Partner}{\textbf{Gesprächspartner: }}
    \clearpage

%%% Quellenverzeichnisse %%%
\literaturverzeichnis
\cleardoublepage

%%% Erklärung %%%
\input{lib/template/_dhbw_erklaerung.tex}

\end{document}
